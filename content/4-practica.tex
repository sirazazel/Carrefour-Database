\newpage

\section{Documents}

\subsection{Estructura d'un usuari}

\begin{singlespace}

  \begin{verbnobox}[\tiny\arabic{VerbboxLineNo}\small\hspace{3ex}]
    [
    {
    "_id": {"\$oid": "6227878c2867551ec8708817"},
    "address": ["Carrer dels Pelats", "76", "07293"],
    "email": "alopecia@gmail.com",
    "name": "Alvaro López",
    "orders": [
              {
              "products": [
                          "622650b1f20aa4641c24569c",
                          "622650b1f20aa4641c2456a8",
                          "622650b1f20aa4641c24569e",
                          "622650b1f20aa4641c2456a0",
                          "622650b1f20aa4641c24569e"
                          ],
              "order_date": "2021-12-23 12:34:21",
              "order_ready": "2021-12-23 18:45:12",
              "shops": ["62266228f20aa4641c2456bd"]
              }
              ],
    "payment": ["Google Pay", "alopecia@gmail.com", "670923128"],
    "username": "alopecia"
    }
    ]
  \end{verbnobox}

\end{singlespace}

Aquest usuari d'exemple veim que ha fet una compra al Carrefour amb 5 productes a 1 tenda, i que va anar a recollir-la a la tenda ja que no té enviament.
Com podem veure, els productes fan referència a un ObjectID d'un document a la col·lecció Productes, i les tendes a un ObjectID de la col·lecció tendes.

\newpage

\subsection{Exemple d'una tenda}

\begin{singlespace}
  
  \begin{verbnobox}[\tiny\arabic{VerbboxLineNo}\small\hspace{3ex}]
    [
      {
        "_id": {"\$oid": "62266228f20aa4641c2456bd"},
        "address": ["C/ Cardenal Rosell", "SN", "07007"],
        "category": ["62265d04f20aa4641c2456b0",
                     "62265d04f20aa4641c2456af", 
                     "62265d04f20aa4641c2456b1", 
                     "62265d04f20aa4641c2456b2", 
                     "62265d04f20aa4641c2456b3", 
                     "62265d04f20aa4641c2456b4"],
        "type": "Hipermercat"
      }
    ]
  \end{verbnobox}

\end{singlespace}

Les tendes tenen varis tipus, hipermercats, gasolineres, magatzems i express. Cada una té diferents categories, que són el que delimita quins productes podem trobar a cada centre.
També tenen l'adreça de la tenda i un id per poder-los referenciar.

\subsection{Exemple d'un producte}

\begin{singlespace}
  \begin{verbnobox}[\tiny\arabic{VerbboxLineNo}\small\hspace{3ex}]
    [
    {
      "_id": {"\$oid": "622650b1f20aa4641c24569d"},
      "name": "Tomàtiga",
      "price": 2.5,
      "type": "Fruit",
      "weight": 0.5
    }
    ]
  \end{verbnobox}
\end{singlespace}

Els productes són documents bastant simples, ja que no referencien res, sempre són referenciats.

\newpage

\subsection{Exemple d'una categoria}

\begin{singlespace}
  \begin{verbnobox}[\tiny\arabic{VerbboxLineNo}\small\hspace{3ex}]
    [
      {
        "_id": {"\$oid": "62265d04f20aa4641c2456b0"},
        "discount": 8,
        "name": "Supermercat",
        "products": ["622650b1f20aa4641c24569c", 
                     "622650b1f20aa4641c24569d",
                     "622650b1f20aa4641c24569e", 
                     "622650b1f20aa4641c24569f", 
                     "622650b1f20aa4641c2456a0", 
                     "622650b1f20aa4641c2456a1", 
                     "622650b1f20aa4641c2456a2"]
      }
    ]
  \end{verbnobox}
\end{singlespace}

Cada categoria té un $n\%$ de descompte que s'aplica a cada compra si ets un client de Carrefour. Aplica aquest percentatge a tots els productes que referenciam des d'aquí.
